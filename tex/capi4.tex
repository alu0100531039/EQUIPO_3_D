%%%%%%%%%%%%%%%%%%%%%%%%%%%%%%%%%%%%%%%%%%%%%%%%%%%%%%%%%%%%%%%%%%%%%%%%%%%%%
% Chapter 4: Conclusiones y Trabajos Futuros 
%%%%%%%%%%%%%%%%%%%%%%%%%%%%%%%%%%%%%%%%%%%%%%%%%%%%%%%%%%%%%%%%%%%%%%%%%%%%%%%

%Como conclusión principal puede decirse que implementando un algoritmo que representa el problema que se quiere resolver, se ahorra tiempo al obtener los resultados, y además éstos pueden ser respresentados en gráficas y tablas para realizar comparaciones con otros experimentos.\\

%Además, el algoritmo no es válido únicamente para el ejemplo en el que se centra el informe, ya que al describir la misma función, simplemente cambiando los datos de entrada puede adaptarse a gran variedad de experimentos con las mismas carácterísticas, con lo cual abarca un campo de estudio mucho más amplio.\\

%Respecto a los propios datos, se llega a la conclusión de  que una distribución de Poisson describe mucho mejor el experimento que algunas de las técnicas más antiguas llevadas a cabo como pueden ser los mencionados 'Código Diamante' y 'Código Triángulo' , ya que los datos obtenidos se aproximan con más exactitud a los reales observados. Este hecho asegura cierta confianza en posibles futuras aplicaciones del experimento en diferentes campos.

%\\

Como conclusion principal puede decirse que implementando un algoritmo que representa el problema que se quiere resolver, se ahorra tiempmo al obtener los resultados y ademas estos pueden ser representados en graficas y tablas para realizar comparaciones con otros experimentos\\
Ademas, el algoritmo no es valido unicamente para el ejemplo en el que se centra el informe, ya que al describir la misma funcion, simplemente cambiando lso datos de entrada puede adaptarse a gran variedad de experimentos con las mismas caracteristicas, con lo cual abarca un campo de estudio mucho mas amplio.\\
Respecto a los propios datos, se llega a la conclusion de que una distribucion de Poisson descrie mucho mejor el eperimento que algunas de las tecnicas mas antiguas llevadas a cabo como pueden ser los mencionados 'Codigo Diamante'  y 'Codigo Triangulo' , ya que los datos obtenidos se aproximan con mas exactitud a los reales observados. Este hecho asegura cierta confianza en posibles futuras aplicaciones del experimento en diferentes campos.